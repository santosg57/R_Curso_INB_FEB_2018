\documentclass{beamer}

% https://matplotlib.org/users/pyplot_tutorial.html#controlling-line-properties

\usetheme{Boadilla}
\usepackage{lmodern}
\usepackage{amsmath}
\usepackage{tikz}
\usepackage{amssymb}
\usetikzlibrary{trees}

% Instituto de Neurobiolog\'ia \\ UNAM - M\'exico

\newcommand\nombre{L. Gonz\'alez-Santos}
\newcommand\depto{Computaci\'on}
\newcommand\instituto{Instituto de Neurobiolog\'ia \\ UNAM}
\newcommand\correo{santosg572@gmail.com}
\newcommand\iniciales{LGS}

\newtheorem{mytheorem}{Theorem}
\newtheorem{mylemma}{Lemma}
\newtheorem{mycorollary}{Corollary}
\theoremstyle{definition}
\newtheorem{mydefinition}{Definition}
\newtheorem{historylabel}{Historycal label}
\newtheorem{myexample}{Example}
\theoremstyle{remark}
\newtheorem{remark}{Remark}

\newcommand{\bR}{{\mathbf{R}}}
\newcommand{\bC}{{\mathbf{C}}}
\newcommand{\bT}{{\mathbf{T}}}
\newcommand{\bZ}{{\mathbf{Z}}}
\newcommand{\myemph}[1]{\alert{#1}}
\newcommand{\essinf}{\mathop{\mathrm{ess\,inf}}}
\renewcommand{\phi}{\varphi}
\newcommand{\eps}{\varepsilon}
\DeclareMathOperator{\adj}{adj}
\DeclareMathOperator{\Range}{Im}
\DeclareMathOperator{\mes}{mes}

\tikzstyle{thinarrow}=[densely dashed,very thin]

\title[Estad\'istica Descriptiva]
{\mbox{Introducci\'on al Lenguaje de Programaci\'on}\\\mbox{R}}
\author[santosg572@gmail.com]{\nombre \\
{\small \correo}}
\institute[\iniciales]{\instituto}
\date{\today}

\hypersetup{pdfkeywords={Hermitian Toeplitz matrices,eigenvectors,asymptotics}}

\AtBeginSection[] {
\begin{frame}{Contenido}
\tableofcontents[currentsection]
\end{frame}
}

%===============================================

\begin{document}

%\begin{frame}[label=titlepage]
%\titlepage

%\end{frame}
%===============================================

%\begin{frame}{Contents}
%\tableofcontents

%\end{frame}

%===============================================
% http://docs.python.org.ar/tutorial/2/introduction.html

%\section{Palabras reservadas en Python}

\begin{frame}{Ejercicio}

\textbf{Ejercicio para enviar hoy martes 6 de marzo.}

\vspace{5mm}

Leer el archivo "datos.txt" que tiene las edades de los participantes de una prueba psicologica.

\vspace{5mm}

Calcular lo siguiente:

\end{frame}

\begin{frame}{Gr\'aficos}

\begin{enumerate}
\item quitar los valores de edad menores que 18 y mayores que 70.
\item Hacer un grafica de la frecuencia de cada edad.
\item Histograma
\item Poligono de frecuencias
\item Distribuci\'on de frecuencias
\item Distribuci\'on de frecuencias relativa
\item Distribuci\'on de frecuencias acumulativa
\item Distribuci\'on de frecuencias acumulativa relativa
\item Barplot
\item BoxPLot
\end{enumerate} 


\end{frame}


\begin{frame}{Medidas de Tendencia Central}

\begin{itemize}
\item Media aritmetica
\item Mediana
\item Moda
\item Skewness
\end{itemize} 
\end{frame}

\begin{frame}{Medidas de Dispersi\'on}

\begin{itemize}
\item Rango
\item Varianza
\item Desviaci\'on Estandard
\item Coeficiente de Variaci\'on
\item Percentiles y Cuantiles
\item Kurtosis
\end{itemize} 
\end{frame}




\end{document}

