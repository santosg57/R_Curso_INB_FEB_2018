\documentclass{beamer}

% https://matplotlib.org/users/pyplot_tutorial.html#controlling-line-properties

\usetheme{Boadilla}
\usepackage{lmodern}
\usepackage{amsmath}
\usepackage{tikz}
\usepackage{amssymb}
\usetikzlibrary{trees}

% Instituto de Neurobiolog\'ia \\ UNAM - M\'exico

\newcommand\nombre{L. Gonz\'alez-Santos}
\newcommand\depto{Computaci\'on}
\newcommand\instituto{Instituto de Neurobiolog\'ia \\ UNAM}
\newcommand\correo{santosg572@gmail.com}
\newcommand\iniciales{LGS}

\newtheorem{mytheorem}{Theorem}
\newtheorem{mylemma}{Lemma}
\newtheorem{mycorollary}{Corollary}
\theoremstyle{definition}
\newtheorem{mydefinition}{Definition}
\newtheorem{historylabel}{Historycal label}
\newtheorem{myexample}{Example}
\theoremstyle{remark}
\newtheorem{remark}{Remark}

\newcommand{\bR}{{\mathbf{R}}}
\newcommand{\bC}{{\mathbf{C}}}
\newcommand{\bT}{{\mathbf{T}}}
\newcommand{\bZ}{{\mathbf{Z}}}
\newcommand{\myemph}[1]{\alert{#1}}
\newcommand{\essinf}{\mathop{\mathrm{ess\,inf}}}
\renewcommand{\phi}{\varphi}
\newcommand{\eps}{\varepsilon}
\DeclareMathOperator{\adj}{adj}
\DeclareMathOperator{\Range}{Im}
\DeclareMathOperator{\mes}{mes}

\tikzstyle{thinarrow}=[densely dashed,very thin]

\title[Distribuciones de Probabilidad]
{\mbox{Introducci\'on al Lenguaje de Programaci\'on}\\\mbox{R}}
\author[santosg572@gmail.com]{\nombre \\
{\small \correo}}
\institute[\iniciales]{\instituto}
\date{\today}

\hypersetup{pdfkeywords={Hermitian Toeplitz matrices,eigenvectors,asymptotics}}

\AtBeginSection[] {
\begin{frame}{Contenido}
\tableofcontents[currentsection]
\end{frame}
}

%===============================================

\begin{document}

\begin{frame}[label=titlepage]
\titlepage

\end{frame}
%===============================================

%\begin{frame}{Contents}
%\tableofcontents

%\end{frame}

%===============================================
% http://docs.python.org.ar/tutorial/2/introduction.html

%\section{Palabras reservadas en Python}

\begin{frame}{Distribuciones de Probabilidad de Variables Discretas}

\textbf{Definici\'on} La distribuci\'on de Probabilidad de una variable aleatoria discreta es una tabla, grafica, formula, que especifica todos los valores posibles de una variable aleatoria discreta junto con sus respectivas probabilidades.

\vspace{3mm}

Si permitimos que la distribuci\'on de probabilidad sea representada por $p(x)$, entonces $p(x) = P(X = x)$ es la probabilidad de la variable aleatoria discreta X que asume el valor $x$.


\end{frame}

\begin{frame}{Distribuciones de Probabilidad de Variables Discretas}

\textbf{Media y Varianza de una Distribuci\'on de Probabilidad Discreta.} 

\[
\mu = \sum x p(x)
\]
\[
\sigma^2 = \sum (x - \mu)^2 p(x) = \sum x^2 p(x) - \mu^2
\]


\end{frame}

\begin{frame}{La Distribuci\'on Binomial }

\begin{itemize}
\item La Distribuci\'on es derivada de un proceso conocido como un ensallo de Berboulli.
\item Swiss mathematician James Bernolli (1954-1705)
\item \textbf{E Proceso de Bernoulli.} Una secuencia de ensayos de Bernoulli forma un proceso de Bernoulli bajo las siguientes condiciones.

\begin{enumerate}
\item Cada ensallo resulta de uno de los dos posibles resultados, mutuamente exclusivos. 
\end{enumerate}

\end{itemize}

\end{frame}






\end{document}

